\documentclass{article}

\usepackage{graphicx}
\graphicspath{ {./img/} }
\usepackage{amsmath}
\usepackage{aezmacros}

% ============================================================================= 
% This is the configuration of the listings for Haskell
% =============================================================================
\usepackage[font=sf,margin={0.5cm,0.5cm}]{caption}

\definecolor{codebackground}{RGB}{255,255,224}
\definecolor{codedeepblue}{RGB}{0,0,200}
\definecolor{codedeepgreen}{RGB}{0,200,0}
\usepackage{listings}
\lstset{ 
  backgroundcolor=\color{codebackground},
  basicstyle=\small\ttfamily,
  extendedchars=true,
  frame=single,
  keepspaces=true,
  keywordstyle=\color{codedeepblue},
  language=Haskell,
  linewidth=14.0cm,
  numbers=none,
  showstringspaces=false,
  stringstyle=\color{codedeepgreen},
  tabsize=2,
  title=\lstname,
  xleftmargin=0.5cm
}
% =============================================================================

\title{Model from Petrie \emph{et al} (2015)}
\date{}
\author{Alexander E. Zarebski}

\begin{document}

\maketitle

\begin{itemize}
\item 
\item 
\item 
\item 
\end{itemize}

\section{Mathematical and computational model}
There are viral species \(A\) and \(B\). Each species has 7 rate parameters. The
rate parameters of the model are represented by the vector \(\theta\).

\begin{equation}
  \theta = (\beta_A,\beta_B,k_A,k_B,\delta_A,\delta_B,p_A,p_B,c_A,c_B,d_A,d_B,\xi_A,\xi_B)
\end{equation}

We can model this with the following type

\begin{lstlisting}[caption={Give me a real caption}, label=parameterTypeDefn]
type Rate = Double

type RateParameters
   = ( Rate
     , Rate
     , Rate
     , Rate
     , Rate
     , Rate
     , Rate
     , Rate
     , Rate
     , Rate
     , Rate
     , Rate
     , Rate
     , Rate)
\end{lstlisting}

Host cells can be in 5 states: \(T\), \(L_i\) and \(I_i\) for \(i=A,B\). The
state of the host system is represented by the vector \(\mathcal{H}\).

\begin{equation}
  \mathcal{H} = (T,L_A,L_B,I_A,I_B)
\end{equation}

The initial condition is known: \(T_0 = 7\times 10^7\) with no latent of
infectious cells. We can model the host state with the following type

\begin{lstlisting}[caption={Give me a real caption}, label=parameterHostDefn]
type HostState = [Double]
\end{lstlisting}

Viral material consists of 4 types: \(V^{TCID}_A\), \(V^{TCID}_B\),
\(V^{RNA}_A\), and \(V^{RNA}_B\). The differential of this system is shown in
the following equations

\begin{equation}
  \begin{aligned}
    \frac{dT}{dt} &= -\beta_A T V^{TCID}_A -\beta_B T V^{TCID}_B \\
    \frac{dL_A}{dt} &= \beta_A T V^{TCID}_A -k_A L_A \\
    \frac{dL_B}{dt} &= \beta_B T V^{TCID}_B -k_B L_B \\
    \frac{dI_A}{dt} &= k_A L_A -\delta_A I_A \\
    \frac{dI_B}{dt} &= k_B L_B -\delta_B I_B 
  \end{aligned}
\end{equation}

The state of the viral system is represented by the vector \(\mathcal{V}\)

\begin{equation}
  \mathcal{V} = (V^{TCID}_A,V^{TCID}_B,V^{RNA}_A,V^{RNA}_B)
\end{equation}

The initial condition is unknown and must be estimated from the data and prior
knowledge of the experimental set up. We represent the intial condition with the
following vector, \(\psi\) to reduce the impact of correlations between the
parameters.

\begin{equation}
  \psi = (V^{TCID}_{comb},V^{RNA}_{comb}/V^{TCID}_{comb},V^{TCID}_b/V^{TCID}_{comb},V^{RNA}_b/V^{RNA}_{comb})
\end{equation}

We can model the viral state with the following type

\begin{lstlisting}[caption={Give me a real caption}, label=parameterViralDefn]
type ViralState = [Double]
\end{lstlisting}

The differential of this system is shown in the following equations

\begin{equation}
  \begin{aligned}
    \frac{dV^{TCID}_A}{dt} &= p_A I_A - c_A V^{TCID}_A - d_A V^{TCID}_A \\
    \frac{dV^{TCID}_B}{dt} &= p_B I_B - c_B V^{TCID}_B - d_B V^{TCID}_B \\
    \frac{dV^{RNA}_A}{dt} &= \xi_A p_A I_A - c_A V^{RNA}_A \\
    \frac{dV^{RNA}_B}{dt} &= \xi_B p_B I_B - c_B V^{RNA}_B 
  \end{aligned}
\end{equation}

The ODE system depends on the \lstinline|RateParameters|, the \lstinline|Time|
and bothe the \lstinline|HostState| and \lstinline|ViralState| which we combine
into a single \lstinline|HostViralState|.

\begin{lstlisting}[caption={Give me a real caption}, label=differentialDefn]
type Time = Double

type HostViralState = [Double]

odeDifferential :: RateParameters -> Time -> HostViralState -> [Double]
odeDifferential (betaA, betaB, kA, kB, deltaA, deltaB, pA, pB, cA, cB, dA, dB, xiA, xiB) _ [hT, hLA, hLB, hIA, hIB, vTCIDA, vTCIDB, vRNAA, vRNAB] =
  [ -betaA * hT * vTCIDA - betaB * hT * vTCIDB
  , betaA * hT * vTCIDA - kA * hLA
  , betaB * hT * vTCIDB - kB * hLB
  , kA * hLA - deltaA * hIA
  , kB * hLB - deltaB * hIB
  , pA * hIA - cA * vTCIDA - dA * vTCIDA
  , pB * hIB - cB * vTCIDB - dB * vTCIDB
  , xiA * pA * hIA - cA * vRNAA
  , xiB * pB * hIB - cB * vRNAB
  ]
\end{lstlisting}

As an example of how this can all be put together, consider the following function

\begin{lstlisting}[caption={Give me a real caption}, label=mainFunc1]
main :: IO ()
main = let ts = linspace 100 (0, 10) :: Vector Time
           initState = [7e7,1,1,1,1,1,1,1,1] :: HostViralState
           params = (1.0e-6,1.0e-6,
                     10.0,10.0,
                     1.0,1.0,
                     1.0,1.0,
                     1.0,1.0,
                     3.12,3.12,
                     1.0,1.0) :: RateParameters
           solMatrix = odeSolve (odeDifferential params) initState ts
           outputFile = "foobar.ssv"
        in saveMatrix outputFile "%.3e" solMatrix
\end{lstlisting}

\begin{figure}
  \centering
  \includegraphics[width=.7\linewidth]{host-fig}
  \caption{Here is a caption}
  \label{fig:host-fig}
\end{figure}

\begin{figure}
  \centering
  \includegraphics[width=.7\linewidth]{virus-fig}
  \caption{Here is a caption}
  \label{fig:virus-fig}
\end{figure}


\end{document}